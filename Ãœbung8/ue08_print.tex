% KETTE. ALSO JEDE SEITE EINZELN ABARBEITEN

\documentclass[11pt]{article}
\pagestyle{empty}
\usepackage[utf8]{inputenc}
\usepackage{a4wide}
\usepackage{amsmath}
\usepackage{amssymb}
\usepackage{amsthm}
\usepackage{german}
\usepackage{mathtools}
\usepackage{amsmath}

\parindent0mm
\sloppy

% Basic data
\newcommand{\VORLESUNG}{ALP I: Funktionale Programmierung}
\newcommand{\STAFF}{A.\ Rudolph und F.\ Formanek}
\newcommand{\ASSIGNMENT}{8}
\newcommand{\HANDOUT}{Samstag, den 14.\ Dezember   2019}
\newcommand{\TUTOR}{Stephanie Hoffmann}
\newcommand{\DELIVER}{bis Montag, den 06.\ Januar 2020, 10:10 Uhr} 


\newcommand{\N}{\mathbb{N}}
\newcommand{\floor}[1]{\lfloor{#1}\rfloor}
\newcommand{\ceil}[1]{\lceil{#1}\rceil}
\newcommand{\half}[1]{\frac{#1}{2}}
\newcommand{\punkte}[1]{{\small{ }(#1 Punkte)}}
\newcommand{\bonuspunke}[1]{{\small{ }(#1 Bonuspunkte)}}

\newcommand{\aufgabe}[1]{\item{\bf #1}}

\begin{document}
\begin{center}
\ASSIGNMENT{}. Aufgabenblatt vom \HANDOUT{} zur Vorlesung 
\vspace*{0.5cm}

{\Large \VORLESUNG{}}

\textbf{Bearbeiter:} \STAFF{}\\
\textbf{Tutor:} \TUTOR\\
\textbf{Tutorium 06}
\vspace*{0.5cm}

{\small Abgabe: \DELIVER{}}
\vspace*{1cm}
\end{center}
\begin{enumerate}
 \aufgabe{Aufgabe}\punkte{24}
\begin{enumerate}
\item
\textit{Behauptung: reverse(reverse xs) = xs}\\
\vspace*{0.5cm}
\textbf{Induktionsanfang:} xs = [ ]\\
reverse(reverse [ ]) $\stackrel{rev.1}{=}$ reverse [ ] $\stackrel{rev.1}{=}$ [ ] $\equiv$ xs

\vspace*{0.5cm}
\textbf{Induktionsvorraussetzung:} für xs = xs' gilt:\\
reverse(reverse xs') = xs'

\vspace*{0.5cm}
\textbf{Indukionsschritt:} Sei xs = (x:xs')
\begin{align*}
&reverse(reverse (x:xs')) &= (x:xs')\\
\stackrel{rev.2}{\equiv} &reverse(reverse xs' ++ [x]) &= (x:xs')\\
\equiv\phantom{.} &(reverse [x]) ++ reverse(reverse xs') &= (x:xs')\\
\stackrel{rev.2}{\equiv} &(reverse [ ] ++ [x]) ++ reverse(reverse xs') &= (x:xs')\\
\stackrel{rev.1}{\equiv} &([ ] ++ [x]) ++ reverse(reverse xs') &= (x:xs')\\
\stackrel{(++).1}{\equiv} &[x] ++ reverse(reverse xs') &= (x:xs')\\
\stackrel{nach IV}{\equiv} &[x] ++ xs' &= (x:xs')\\
\equiv\phantom{.} &(x:xs') &= (x:xs')\\
\end{align*}
\vspace*{0.5cm}
\textbf{Das bedeutet, dass die Behauptung für alle xs (endliche Listen) gilt.}
\newpage
\vspace*{0.5cm}
\item
\textit{Behauptung: reverse(xs ++ ys) = reverse ys ++ reverse xs}\\
\vspace*{0.5cm}
\textbf{Induktionsanfang:} xs = [ ]
\begin{alignat*}{2}
&reverse([] ++ ys) &&= reverse\phantom{.}ys \phantom{.}++\phantom{.} reverse\phantom{.}[]\\
\stackrel{rev.1}{\equiv} &reverse([] ++ ys) &&= reverse\phantom{.} ys \phantom{.}++\phantom{.} []\\
\stackrel{(++).1}{\equiv} &reverse\phantom{.} ys &&= reverse\phantom{.} ys
\end{alignat*}

\vspace*{0.5cm}
\textbf{Induktionsvorraussetzung:} für xs = xs' gilt:\\
reverse(xs' ++ ys) = reverse ys ++ reverse xs'

\vspace*{0.5cm}
\textbf{Indukionsschritt:} Sei xs = (x:xs')
\begin{alignat*}{2}
&reverse((x:xs')\phantom{.}++\phantom{.}ys) &&= reverse\phantom{.}ys \phantom{.}++\phantom{.}reverse(x:xs')\\
\stackrel{rev.2}{\equiv} &reverse((x:xs')\phantom{.}++\phantom{.}ys) &&= reverse \phantom{.} ys\phantom{.}++\phantom{.}(reverse\phantom{.}xs'\phantom{.}++\phantom{.}[x])\\
\stackrel{(++).2}{\equiv} &reverse(x:(xs'++ys)) &&= reverse\phantom{.}ys\phantom{.}++\phantom{.}(reverse\phantom{.}xs'\phantom{.}++\phantom{.}[x])\\
\stackrel{rev.2}{\equiv}&reverse(xs'++ys)\phantom{.}++\phantom{.}[x]&&=reverse\phantom{.}ys\phantom{.}++\phantom{.}(reverse\phantom{.}xs'\phantom{.}++\phantom{.}[x])\\
\stackrel{nach IV}{\equiv} &reverse\phantom{.}ys\phantom{.}++\phantom{.}reverse\phantom{.}xs'\phantom{.}++\phantom{.}[x] &&= reverse\phantom{.}ys++\phantom{.}(reverse\phantom{.}xs'\phantom{.}++\phantom{.}[x])\\
\equiv \phantom{.}&reverse\phantom{.}ys\phantom{.}++\phantom{.}reverse\phantom{.}xs'\phantom{.}++\phantom{.}[x] &&= reverse\phantom{.}ys\phantom{.}++\phantom{.}reverse\phantom{.}xs'\phantom{.}++\phantom{.}[x]
\end{alignat*}


\vspace*{0.5cm}
\textbf{Das bedeutet, dass die Behauptung für alle xs (endliche Listen) gilt.}

\vspace*{0.5cm}
\item
\textit{Behauptung: elem a (xs ++ ys) = elem a xs $||$ elem a ys}\\
\textbf{Induktionsanfang:} xs = []
\begin{alignat*}{3}
&elem\phantom{.}a\phantom{.}([]\phantom{.}++\phantom{.}ys) &&= elem\phantom{.}a\phantom{.}[]\phantom{.}&&||\phantom{.}elem\phantom{.}a\phantom{.}ys\\
\stackrel{(++).1}{\equiv}&elem\phantom{.}a\phantom{.}ys &&= elem\phantom{.}a\phantom{.}[]\phantom{.}&&||\phantom{.}elem\phantom{.}a\phantom{.}ys\\
\stackrel{elem.1}{\equiv}&elem\phantom{.}a\phantom{.}ys&&=False\phantom{.}&&||\phantom{.}elem\phantom{.}a\phantom{.}ys\\
\equiv \phantom{.}&elem\phantom{.}a\phantom{.}ys&&=elem\phantom{.}a\phantom{.}ys &&\
\end{alignat*}


\vspace*{0.5cm}
\textbf{Induktionsvorraussetzung:} für xs = xs' gilt:\\
elem a (xs' ++ ys) = elem a xs' $||$ elem a ys
\newpage
\textbf{Indukionsschritt:} Sei xs = (x:xs')\\
\phantom{Krie}\textbf{Fall 1 (x immer ungleich y):}
\begin{alignat*}{3}
&elem\phantom{.}a\phantom{.}((x:xs')\phantom{.}++\phantom{.}ys)&&=elem\phantom{.}a\phantom{.}(x:xs')\phantom{.}&&||\phantom{.}elem\phantom{.}a\phantom{.}ys\\
\stackrel{(++).2}{\equiv}&elem\phantom{.}a\phantom{.}(x:(xs'\phantom{.}++\phantom{.}ys))&&=elem\phantom{.}a\phantom{.}(x:xs')\phantom{.}&&||\phantom{.}elem\phantom{.}a\phantom{.}ys\\
\stackrel{elem.3}{\equiv}&elem\phantom{.}a\phantom{.}(x:(xs'\phantom{.}++\phantom{.}ys))&&=elem\phantom{.}a\phantom{.}ys\phantom{.}&&||\phantom{.}elem\phantom{.}a\phantom{.}ys\\
\equiv\phantom{.}&elem\phantom{.}a\phantom{.}(x:(xs'\phantom{.}++\phantom{.}ys))&&=elem\phantom{.}a\phantom{.}ys&&\\
\equiv\phantom{.}&elem\phantom{.}a\phantom{.}[x]\phantom{.}||\phantom{.}elem\phantom{.}a\phantom{.}(xs'\phantom{.}++\phantom{.}ys)&&=elem\phantom{.}a\phantom{.}ys&&\\
\stackrel{nach IV}{\equiv}&elem\phantom{.}a\phantom{.}[x]\phantom{.}||\phantom{.}(elem\phantom{.}a\phantom{.}xs'\phantom{.}||\phantom{.}elem\phantom{.}a\phantom{.}ys)&&=elem\phantom{.}a\phantom{.}ys&&\\
\equiv\phantom{.}&elem\phantom{.}a\phantom{.}(x:[])\phantom{.}||\phantom{.}(elem\phantom{.}a\phantom{.}xs'\phantom{.}||\phantom{.}elem\phantom{.}a\phantom{.}ys)&&=elem\phantom{.}a\phantom{.}ys&&\\
\stackrel{elem.1}{\equiv}&False\phantom{.}||\phantom{.}(elem\phantom{.}a\phantom{.}xs'\phantom{.}||\phantom{.}elem\phantom{.}a\phantom{.}ys)&&=elem\phantom{.}a\phantom{.}ys&&\\
\equiv\phantom{.}&elem\phantom{.}a\phantom{.}xs'\phantom{.}||\phantom{.}elem\phantom{.}a\phantom{.}ys&&=elem\phantom{.}a\phantom{.}ys&&\\
&\text{\textbf{Liefert kein eindeutiges Ergebnis}}
\end{alignat*}
\newline
\phantom{Krie}\textbf{Fall 2 (x==y):}
\begin{alignat*}{3}
&elem\phantom{.}a\phantom{.}((x:xs')\phantom{.}++\phantom{.}ys)&&=elem\phantom{.}a\phantom{.}(x:xs')\phantom{.}&&||\phantom{.}elem\phantom{.}a\phantom{.}ys\\
\stackrel{(++).2}{\equiv}&elem\phantom{.}a\phantom{.}(x:(xs'\phantom{.}++\phantom{.}ys))&&=elem\phantom{.}a\phantom{.}(x:xs')\phantom{.}&&||\phantom{.}elem\phantom{.}a\phantom{.}ys\\
\stackrel{elem.2}{\equiv}&elem\phantom{.}a\phantom{.}(x:(xs'\phantom{.}++\phantom{.}ys))&&=True\phantom{.}&&||\phantom{.}elem\phantom{.}a\phantom{.}ys\\
\equiv\phantom{.}&elem\phantom{.}a\phantom{.}(x:(xs'\phantom{.}++\phantom{.}ys))&&=True &&\\
\equiv\phantom{.}&elem\phantom{.}a\phantom{.}[x]\phantom{.}||\phantom{.}elem\phantom{.}a\phantom{.}(xs'\phantom{.}++\phantom{.}ys)&&=True &&\\
\stackrel{elem.2}{\equiv}&True\phantom{.}||\phantom{.}elem\phantom{.}a\phantom{.}(xs'\phantom{.}++\phantom{.}ys)&&=True&&\\
\equiv\phantom{.}&True&&=True
\end{alignat*}

\vspace*{0.5cm}
\textbf{Das bedeutet, dass die Behauptung für (fast) alle xs (endliche Listen) gilt.}

\vspace*{0.5cm}
\item
\textit{Behauptung: (takeWhile p xs) ++ (dropWhile p xs) = xs}\\
\textbf{Induktionsanfang:} xs = [ ]
\begin{alignat*}{2}
&(takeWhile\phantom{.}p\phantom{.}[])\phantom{.}++\phantom{.}(dropWhile\phantom{.}p\phantom{.}[])&&=[]\\
\stackrel{takeW.1}{\equiv}&[]\phantom{.}++\phantom{.}(dropWhile\phantom{.}p\phantom{.}[])&&=[]\\
\stackrel{dropW.1}{\equiv}&[]\phantom{.}++\phantom{.}[]&&=[]\\
\stackrel{(++).1}{\equiv}&[]&&=[]
\end{alignat*}

\newpage
\textbf{Induktionsvorraussetzung:} für xs = xs' gilt:\\
(takeWhile p xs') ++ (dropWhile p xs') = xs'

\vspace*{0.5cm}
\textbf{Indukionsschritt:} Sei xs = (x:xs')
\begin{alignat*}{2}
&(takeWhile\phantom{.}p\phantom{.}(x:xs'))\phantom{.}++\phantom{.}(dropWhile\phantom{.}p\phantom{.}(x:xs'))&&=(x:xs')\\
\stackrel{takeW.2}{\equiv}&(x:(takeWhile\phantom{.}p\phantom{.}xs'))\phantom{.}++\phantom{.}(dropWhile\phantom{.}p\phantom{.}(x:xs'))&&=(x:xs')\\
\stackrel{dropW.2}{\equiv}&(x:(takeWhile\phantom{.}p\phantom{.}xs'))\phantom{.}++\phantom{.}(dropWhile\phantom{.}p\phantom{.}xs')&&=(x:xs')\\
\stackrel{(++).2}{\equiv}&x:((takeWhile\phantom{.}p\phantom{.}xs')\phantom{.}++\phantom{.}(dropWhile\phantom{.}p\phantom{.}xs')&&=(x:xs')\\
\stackrel{nach IV}{\equiv}&(x:xs')&&=(x:xs')
\end{alignat*}

\vspace*{0.5cm}
\textbf{Das bedeutet, dass die Behauptung für alle xs (endliche Listen) gilt.}

\vspace*{0.5cm}
\item
\textit{Behauptung: map (f . g) xs = map f xs . map g xs}\\
\textbf{Induktionsanfang:} xs = [ ]
\begin{alignat*}{2}
&map\phantom{.}(f\phantom{.}.\phantom{.}g)\phantom{.}[]&&=map\phantom{.}f\phantom{.}[]\phantom{.}.\phantom{.}map\phantom{.}g\phantom{.}[]\\
\stackrel{map.1}{\equiv}&[]&&=[]
\end{alignat*}

\vspace*{0.5cm}
\textbf{Induktionsvorraussetzung:} für xs = xs' gilt:\\
map (f . g) xs' = map f xs' . map g xs'

\vspace*{0.5cm}
\textbf{Indukionsschritt:} Sei xs = (x:xs')
\begin{alignat*}{2}
&map\phantom{.}(f\phantom{.}.\phantom{.}g)\phantom{.}(x:xs')&&=map\phantom{.}f\phantom{.}(x:xs')\phantom{.}.\phantom{.}map\phantom{.}g\phantom{.}(x:xs')\\
\stackrel{map.2}{\equiv}&g(f(x)):map(f\phantom{.}.\phantom{.}g)\phantom{.}xs'&&=g(f(x)):(map\phantom{.}f\phantom{.}xs'\phantom{.}.\phantom{.}map\phantom{.}g\phantom{.}xs')\\
\stackrel{nach IV}{\equiv}&g(f(x)):(map\phantom{.}f\phantom{.}xs'\phantom{.}.\phantom{.}map\phantom{.}g\phantom{.}xs')&&=g(f(x)):(map\phantom{.}f\phantom{.}xs'\phantom{.}.\phantom{.}map\phantom{.}g\phantom{.}xs')
\end{alignat*}

\vspace*{0.5cm}
\textbf{Das bedeutet, dass die Behauptung für alle xs (endliche Listen) gilt.}
\newpage
\vspace*{0.5cm}

\item
\textit{Behauptung: map f . concat = concat . map(map f)}\\
\textbf{Induktionsanfang:} xs = [ ]
\begin{alignat*}{2}
&map\phantom{.}f\phantom{.}.\phantom{.}concat\phantom{.}[]&&=concat\phantom{.}.\phantom{.}map(map\phantom{.}f)\phantom{.}[]\\
\stackrel{map.1}{\equiv}&concat\phantom{.}[]&&=concat\phantom{.}.\phantom{.}map(map\phantom{.}f)\phantom{.}[]\\
\stackrel{concat.1}{\equiv}&concat\phantom{.}[]&&=(foldr\phantom{.}(++)\phantom{.}[]\phantom{.}[])\phantom{.}.\phantom{.}map(map\phantom{.}f)\\
\stackrel{foldr.1}{\equiv}&concat\phantom{.}[]&&=[]\phantom{.}.\phantom{.}map(map\phantom{.}f)\\
\stackrel{map.1}{\equiv}&concat\phantom{.}[]&&=[]\\
\stackrel{concat.1}{\equiv}&[]&&=[]
\end{alignat*}

\vspace*{0.5cm}
\textbf{Induktionsvorraussetzung:} für xs = xs' gilt:\\
map f . concat xs' = concat . map(map f) xs'

\vspace*{0.5cm}
\textbf{Indukionsschritt:} Sei xs = (x:xs')\\
/
\end{enumerate}
	\aufgabe{Aufgabe}\punkte{4}\\
\vspace*{0.5cm}
\textit{Behauptung: length(powset xs) = $2^{(length(xs))}$}\\
\textbf{Induktionsanfang:} xs = [ ]
\begin{alignat*}{2}
&length(powset\phantom{.}[])&&=2^{(length\phantom{.}[])}\\
\stackrel{e.1}{\equiv}&length(powset\phantom{.}[])&&=2^1\\
\stackrel{pow.1}{\equiv}&length([[]])&&=2^1\\
\stackrel{e.1}{\equiv}&1&&=2^1\\
\equiv\phantom{.}&1&&=1
\end{alignat*}

\vspace*{0.5cm}
\textbf{Induktionsvorraussetzung:} für xs = xs' gilt:\\
length(powset xs') = $2^{(length(xs'))}$
\newpage
\vspace*{0.5cm}
\textbf{Indukionsschritt:} Sei xs = (x:xs')
\begin{alignat*}{2}
&length(powset\phantom{.}(x:xs'))&&=2^{(length(x:xs'))}\\
\stackrel{pow.2}{\equiv}&length(powset\phantom{.}xs'\phantom{.}++\phantom{.}[x:ys\phantom{.}|\phantom{.}ys \leftarrow powset\phantom{.}xs'])&&=2^{(length(x:xs'))}\\
\equiv\phantom{.}&length(powset\phantom{.}xs')\phantom{.}+\phantom{.}(length\phantom{.}[x:ys\phantom{.}|\phantom{.}ys \leftarrow powset\phantom{.}xs'])&&=2^{(length(x:xs'))}\\
\stackrel{nach IV}{\equiv}&2^{(length\phantom{.}xs')}\phantom{.}+\phantom{.}(length\phantom{.}[x:ys\phantom{.}|\phantom{.}ys\leftarrow powset\phantom{.}xs'])&&=2^{(length(x:xs'))}\\
\stackrel{e.1}{\equiv}\phantom{.}&2^{(length\phantom{.}xs')}\phantom{.}+\phantom{.}(length(powset\phantom{.}xs')\phantom{.}+\phantom{.}1)&&=2^{(length(x:xs'))}\\
\stackrel{nach IV}{\equiv}&2^{(length\phantom{.}xs')}\phantom{.}+\phantom{.}2^{((length\phantom{.}xs')\phantom{.}+\phantom{.}1)}&&=2^{(length(x:xs'))}\\
\equiv\phantom{.}&2^{(length\phantom{.}xs')}\phantom{.}*\phantom{.}2^1&&=2^{(length\phantom{.}xs')}\phantom{.}*\phantom{.}2^1
\end{alignat*}

\vspace*{0.5cm}
\textbf{Das bedeutet, dass die Behauptung für alle xs (endliche Listen) gilt.}

\aufgabe{Aufgabe}\bonuspunke{4}\\
\vspace*{0.5cm}
\textit{Behauptung: sumLeaves t = sumNodes t + 1}\\
\textbf{Induktionsanfang:} Sei t = (Leaf x)
\begin{alignat*}{2}
&sumLeaves\phantom{.}(Leaf\phantom{.}x)&&=sumNodes\phantom{.}(Leaf\phantom{.}x)\phantom{.}+\phantom{.}1\\
\stackrel{sumL.1}{\equiv}&1&&=sumNodes\phantom{.}(Leaf\phantom{.}x)\phantom{.}+\phantom{.}1\\
\stackrel{sumN.1}{\equiv}&1 &&= 0\phantom{.}+\phantom{.}1\\
\equiv\phantom{.}&1&&=1
\end{alignat*}

\vspace*{0.5cm}
\textbf{Induktionsvorraussetzung:} für t = (Node x lt rt) gilt:\\
sumLeaves lt = sumNodes lt + 1\\
\phantom{Kriegel }und\\
sumLeaves rt = sumNodes rt + 1

\vspace*{0.5cm}
\textbf{Indukionsschritt:} Sei t = (Node x lt rt)
\begin{alignat*}{2}
&sumLeaves\phantom{.}(Node\phantom{.}x\phantom{.}lt\phantom{.}rt)&&=sumNodes\phantom{.}(Node\phantom{.}x\phantom{.}lt\phantom{.}rt)\phantom{.}+\phantom{.}1\\
\stackrel{sumL.2}{\equiv}&sumLeaves\phantom{.}lt\phantom{.}+\phantom{.}sumLeaves\phantom{.}rt&&=sumNodes\phantom{.}(Node\phantom{.}x\phantom{.}lt\phantom{.}rt)\phantom{.}+\phantom{.}1\\
\stackrel{sumN.2}{\equiv}&sumLeaves\phantom{.}lt\phantom{.}+\phantom{.}sumLeaves\phantom{.}rt&&=1\phantom{.}+\phantom{.}sumNodes\phantom{.}lt\phantom{.}+\phantom{.}sumNodes\phantom{.}rt\phantom{.}+\phantom{.}1\\
\stackrel{nach IV}{\equiv}&sumNodes\phantom{.}lt\phantom{.}+\phantom{.}1+\phantom{.}sumNodes\phantom{.}rt\phantom{.}+\phantom{.}1&&=1\phantom{.}+\phantom{.}sumNodes\phantom{.}lt\phantom{.}+\phantom{.}sumNodes\phantom{.}rt\phantom{.}+\phantom{.}1\\
\equiv\phantom{.}&sumNodes\phantom{.}lt\phantom{.}+\phantom{.}sumNodes\phantom{.}rt\phantom{.}+\phantom{.}2&&=sumNodes\phantom{.}lt\phantom{.}+\phantom{.}sumNodes\phantom{.}rt\phantom{.}+\phantom{.}2
\end{alignat*}

\vspace*{0.5cm}
\textbf{Das bedeutet, dass die Behauptung für alle t (endliche Binärbäume) gilt.}

\aufgabe{Aufgabe}\bonuspunke{6}\\
\vspace*{0.5cm}
\textit{Behauptung: sum\phantom{.}.\phantom{.}tree2List = sumTree}\\
\textbf{Induktionsanfang 1.1:} Sei t = Nil
\begin{alignat*}{2}
&sum\phantom{.}.\phantom{.}tree2List\phantom{.}Nil &&= sumTree\phantom{.}Nil\\
\stackrel{sumT.1}{\equiv}&sum\phantom{.}.\phantom{.}tree2List\phantom{.}Nil &&= 0\\
\stackrel{t2L.1}{\equiv}&sum\phantom{.}[]&&=0\\
\stackrel{sum.1}{\equiv}&0&&=0
\end{alignat*}
\textbf{Induktionsanfang 1.2:} Sei t = (Leaf x)
\begin{alignat*}{2}
&sum\phantom{.}.\phantom{.}tree2List\phantom{.}(Leaf\phantom{.}x)&&=sumTree\phantom{.}(Leaf\phantom{.}x)\\
\stackrel{sumT.2}{\equiv}&sum\phantom{.}.\phantom{.}tree2List\phantom{.}(Leaf\phantom{.}x)&&=x\\
\stackrel{t2L.2}{\equiv}&sum\phantom{.}[x]&&=x\\
\equiv\phantom{.}&sum(x:[])&&=x\\
\stackrel{sum.2}{\equiv}&x\phantom{.}+\phantom{.}sum\phantom{.}[]&&=x\\
\stackrel{sum.1}{\equiv}&x\phantom{.}+\phantom{.}0&&=x\\
\equiv\phantom{.}&x&&=x
\end{alignat*}

\vspace*{0.5cm}
\textbf{Induktionsvorraussetzung:} für t = (Node x lt rt) gilt:\\
sum\phantom{.}.\phantom{.}tree2List lt = sumTree lt\\
\phantom{Kriegel }und\\
sum\phantom{.}.\phantom{.}tree2List rt = sumTree rt\\


\textbf{Indukionsschritt:} Sei t = (Node x lt rt)
\begin{alignat*}{2}
&sum\phantom{.}.\phantom{.}list2Tree\phantom{.}(Node\phantom{.}x\phantom{.}lt\phantom{.}rt)&&=sumTree\phantom{.}(Node\phantom{.}x\phantom{.}lt\phantom{.}rt)\\
\stackrel{sumT.3}{\equiv}&sum\phantom{.}.\phantom{.}list2Tree\phantom{.}(Node\phantom{.}x\phantom{.}lt\phantom{.}rt)&&=x\phantom{.}+\phantom{.}sumTree\phantom{.}lt\phantom{.}+\phantom{.}sumTree\phantom{.}rt\\
\stackrel{l2T.3}{\equiv}&sum(tree2List\phantom{.}lt\phantom{.}++\phantom{.}[x]\phantom{.}++\phantom{.}tree2List\phantom{.}rt)&&=x\phantom{.}+\phantom{.}sumTree\phantom{.}lt\phantom{.}+\phantom{.}sumTree\phantom{.}rt\\
\equiv\phantom{.}&sum\phantom{.}[x]\phantom{.}+\phantom{.}(sum(tree2List\phantom{.}lt))\phantom{.}+\phantom{.}(sum(tree2List\phantom{.}rt))&&=x\phantom{.}+\phantom{.}sumTree\phantom{.}lt\phantom{.}+\phantom{.}sumTree\phantom{.}rt\\
\stackrel{nach IV}{\equiv}&sum\phantom{.}[x]\phantom{.}+\phantom{.}sumTree\phantom{.}lt\phantom{.}+\phantom{.}sumTree\phantom{.}rt&&=x\phantom{.}+\phantom{.}+sumTree\phantom{.}lt\phantom{.}+\phantom{.}sumTree\phantom{.}rt\\
\stackrel{nach IA}{\equiv}&x\phantom{.}+\phantom{.}sumTree\phantom{.}lt\phantom{.}+\phantom{.}sumTree\phantom{.}rt&&=x\phantom{.}+\phantom{.}sumTree\phantom{.}lt\phantom{.}+\phantom{.}sumTree\phantom{.}rt
\end{alignat*}

\vspace*{0.5cm}
\textbf{Das bedeutet, dass die Behauptung für alle t (endliche Binärbäume) gilt.}
\end{enumerate}
\end{document}
