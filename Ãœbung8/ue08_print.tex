% KETTE. ALSO JEDE SEITE EINZELN ABARBEITEN

\documentclass[11pt]{article}
\pagestyle{empty}
\usepackage[utf8]{inputenc}
\usepackage{a4wide}
\usepackage{amsmath}
\usepackage{amssymb}
\usepackage{amsthm}
\usepackage{german}
\usepackage{mathtools}

\parindent0mm
\sloppy

% Basic data
\newcommand{\VORLESUNG}{ALP I: Funktionale Programmierung}
\newcommand{\STAFF}{A.\ Rudolph und F.\ Formanek}
\newcommand{\ASSIGNMENT}{8}
\newcommand{\HANDOUT}{Samstag, den 14.\ Dezember   2019}
\newcommand{\TUTOR}{Stephanie Hoffmann}
\newcommand{\DELIVER}{bis Montag, den 06.\ Januar 2020, 10:10 Uhr} 


\newcommand{\N}{\mathbb{N}}
\newcommand{\floor}[1]{\lfloor{#1}\rfloor}
\newcommand{\ceil}[1]{\lceil{#1}\rceil}
\newcommand{\half}[1]{\frac{#1}{2}}
\newcommand{\punkte}[1]{{\small{ }(#1 Punkte)}}

\newcommand{\aufgabe}[1]{\item{\bf #1}}

\begin{document}
\begin{center}
\ASSIGNMENT{}. Aufgabenblatt vom \HANDOUT{} zur Vorlesung 
\vspace*{0.5cm}

{\Large \VORLESUNG{}}

\textbf{Bearbeiter:} \STAFF{}\\
\textbf{Tutor:} \TUTOR\\
\textbf{Tutorium 06}
\vspace*{0.5cm}

{\small Abgabe: \DELIVER{}}
\vspace*{1cm}
\end{center}
\begin{enumerate}
 \aufgabe{Aufgabe}\punkte{24}
\begin{enumerate}
\item
\textit{Behauptung: reverse(reverse xs) = xs}\\
Induktion über Liste xs der Länge n
\vspace*{0.5cm}
\newline
\textbf{Induktionsanfang:} xs = [ ]\\
reverse(reverse [ ]) $\stackrel{rev.1}{=}$ reverse [ ] $\stackrel{rev.1}{=}$ [ ]

\vspace*{0.5cm}
\textbf{Induktionsvorraussetzung:} für xs = xs' gilt:\\
reverse(reverse xs') = xs'

\vspace*{0.5cm}
\textbf{Indukionsschritt:} Sei xs = (x:xs')\\
reverse(reverse (x:xs')) $\stackrel{rev.2}{=}$ reverse(reverse xs' ++ [x])\\ 
$\equiv$ (reverse [x]) ++ reverse (reverse xs')\\ $\stackrel{rev.2}
{\leftrightarrow}$ (reverse ([ ]) ++ [x]) ++ reverse(reverse xs') $\stackrel{rev.1}{\leftrightarrow}$\\
([ ] ++ [x]) ++ reverse(reverse xs') $\stackrel{(++).1}{=}$\\
\ [x] ++ reverse(reverse xs') $\stackrel{nach IV}{=}$\\
\textbf{[x] ++ xs' $\equiv$ (x:xs')}

\vspace*{0.5cm}
\textbf{Das bedeutet, dass die Behauptung für alle xs (endliche Listen) gilt.}

\vspace*{0.5cm}
\item
\textit{Behauptung: reverse(xs ++ ys) = reverse ys ++ reverse xs}
\vspace*{0.5cm}
\newline
\textbf{Induktionsanfang:} xs = [ ]\\
reverse([] ++ ys) = reverse ys ++ reverse [] $\stackrel{rev.1}{=}$\\
reverse([] ++ ys) = reverse ys ++ [] $\stackrel{(++).1}{=}$\\
reverse ys = reverse ys

\vspace*{0.5cm}
\textbf{Induktionsvorraussetzung:} für xs = xs' gilt:\\
reverse(xs' ++ ys) = reverse ys ++ reverse xs'

\vspace*{0.5cm}
\textbf{Indukionsschritt:} Sei xs = (x:xs')\\
reverse((x:xs') ++ ys) = reverse ys ++ reverse(x:xs') $\stackrel{rev.2}{=}$\\
reverse((x:xs') ++ ys) = reverse ys ++ (reverse xs' ++ [x]) $\stackrel{(++).2}{=}$\\
reverse(x:(xs'++ys)) = reverse ys ++ (reverse xs' ++ [x]) $\stackrel{rev.2}{=}$\\
reverse(xs'++ys) ++ [x] = reverse ys ++ (reverse xs' ++[x]) $\stackrel{nach IV}{=}$\\
reverse ys ++ reverse xs' ++ [x] = reverse ys ++ (reverse xs' ++ [x]) $\equiv$\\
\textbf{reverse ys ++ reverse xs' ++ [x] = reverse ys ++ reverse xs' ++ [x]}

\vspace*{0.5cm}
\textbf{Das bedeutet, dass die Behauptung für alle xs (endliche Listen) gilt.}

\vspace*{0.5cm}
\item
\textit{Behauptung: elem a (xs ++ ys) = elem a xs $||$ elem a ys}
\vspace*{0.5cm}
\newline
\textbf{Induktionsanfang:} xs = [ ]\\
elem a ([ ] ++ ys) = elem a [ ] $||$ elem a ys $\stackrel{(++).1}{=}$\\
elem a ys = elem a [] $||$ elem a ys $\stackrel{elem.1}{=}$\\
elem a ys = False $||$ elem a ys $\equiv$\\
elem a ys = elem a ys

\vspace*{0.5cm}
\textbf{Induktionsvorraussetzung:} für xs = xs' gilt:\\
elem a (xs' ++ ys) = elem a xs' $||$ elem a ys

\vspace*{0.5cm}
\textbf{Indukionsschritt:} Sei xs = (x:xs')\\
elem a ((x:xs') ++ ys) = elem a (x:xs') $||$ elem a ys $\stackrel{(++).1}{=}$\\
elem a (x:(xs' ++ ys)) = elem a (x:xs') $||$ elem a ys $\stackrel{elem.3}{=}$\\
elem a (x:(xs' ++ ys)) = elem a ys $||$ elem a ys $\equiv$\\
elem a (x:(xs' ++ ys)) = elem a ys $\equiv$\\
elem a [x] $||$ elem a (xs' ++ ys) = elem a ys $\stackrel{nach IV}{=}$\\
elem a [x] $||$ (elem a xs' $||$ elem a ys) = elem a ys $\equiv$\\
elem a (x:[]) $||$ (elem a xs' $||$ elem a ys) = elem a ys $\stackrel{elem.1}{=}$\\
False $||$ (elem a xs' $||$ elem a ys) = elem a ys $\equiv$\\
elem a xs' $||$ elem a ys = elem a ys\\

TODO: Überarbeiten. Ergebnis ist falsch


\vspace*{0.5cm}
\textbf{Das bedeutet, dass die Behauptung für alle xs (endliche Listen) gilt.}

\vspace*{0.5cm}
\item
\textit{Behauptung: (takeWhile p xs) ++ (dropWhile p xs) = xs}
\vspace*{0.5cm}
\newline
\textbf{Induktionsanfang:} xs = [ ]\\
(takeWhile p [ ]) ++ (dropWhile p [ ]) = [ ] $\stackrel{takeW.1}{=}$\\
\ [ ] ++ (dropWhile p [ ]) = [ ] $\stackrel{dropW.1}{=}$\\
\ [ ] ++ [ ] = [ ] $\stackrel{(++).1}{=}$\\
\ [ ] = [ ]


\vspace*{0.5cm}
\textbf{Induktionsvorraussetzung:} für xs = xs' gilt:\\
(takeWhile p xs') ++ (dropWhile p xs') = xs'

\vspace*{0.5cm}
\textbf{Indukionsschritt:} Sei xs = (x:xs')\\
(takeWhile p (x:xs')) ++ (dropWhile p (x:xs')) = (x:xs') $\stackrel{takeW.2}{=}$\\
(x:(takeWhile p xs')) ++ (dropWhile p (x:xs')) = (x:xs') $\stackrel{dropW.2}{=}$\\
(x:(takeWhile p xs')) ++ (dropWhile p xs') = (x:xs') $\stackrel{(++).2}{=}$\\
x:((takeWhile p xs') ++ (dropWhile p xs') = (x:xs') $\stackrel{nach IV}{=}$\\
\textbf{(x:xs') = (x:xs')}

\vspace*{0.5cm}
\textbf{Das bedeutet, dass die Behauptung für alle xs (endliche Listen) gilt.}

\vspace*{0.5cm}
\item
\textit{Behauptung: map (f . g) xs = map f xs . map g xs}

\vspace*{0.5cm}
\textbf{Induktionsanfang:} xs = [ ]\\
map (f . g) [ ] = map f [ ] . map g [ ] $\stackrel{map.1}{=}$\\
\ [ ] = [ ]


\vspace*{0.5cm}
\textbf{Induktionsvorraussetzung:} für xs = xs' gilt:\\
map (f . g) xs' = map f xs' . map g xs'

\vspace*{0.5cm}
\textbf{Indukionsschritt:} Sei xs = (x:xs')\\
map (f . g) (x:xs') = map f (x:xs') . map g (x:xs') $\stackrel{map.2}{=}$\\
g(f(x)):map(f . g) xs' = g(f(x)):(map f xs' . map g xs') $\stackrel{nach IV}{=}$\\
\textbf{g(f(x)):(map f xs' . map g xs') = g(f(x)):(map f xs' . map g xs')}

\vspace*{0.5cm}
\textbf{Das bedeutet, dass die Behauptung für alle xs (endliche Listen) gilt.}
\end{enumerate}
\end{enumerate}
\end{document}
